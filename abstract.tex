\begin{abstract}
   One of the earliest problems in Artificial Intelligence has been to make computers become good at playing games. Most of the early work in this area was centred around traditional AI “search” algorithms like minimax search with alpha beta pruning. This was followed by more "machine learning" centric approaches which enabled agents to learn from experience. In this current work, we propose a technique that first learns the rules of a game and then learns a strategy on top of this by experiencing the game itself, thereby emulating a more "human-learning" centred behaviour. We have used Connect-4 as the game of our choice. This is an interesting technique because (1) it is similar to how humans learn to play games and (2) this method should be generalizable to different games including ones where search and other techniques perform poorly. Hence, with this approach it should be easy to extend the same system to learn to play and win a new game (with a grid-based structure) like Tic-Tac-Toe, Connect 5 etc.

\end{abstract}